\documentclass[xcolor=dvipsnames]{beamer}

\usepackage[utf8]{inputenc}
\usepackage[ngerman]{babel}
\usepackage{graphicx}
\usepackage[svgpath=svg/,clean]{svg} %Has to be loaded after graphicx
\usepackage{tikz}
\PassOptionsToPackage{fleqn}{amsmath}       % math environments and more by the AMS
    \usepackage{amsmath}

%Information
\title[Eine empirische Studie]{Boundaries of Polyhedral Compilation}
\author{Stefan Huber}
\institute{
    Fakultät für Informatik und Mathematik\\
    Universität Passau
}
\date{\today}
\subject{Informatik}

%Configuration
\usetheme{Rochester}
\usecolortheme{whale}
\beamertemplatenavigationsymbolsempty

\AtBeginSection[]{
    \begin{frame}
        \frametitle{Inhaltsangabe}
        \tableofcontents[currentsection]
    \end{frame}
}

%custom colors
\definecolor{defaultColor}{HTML}{1E90FF}
\definecolor{llvmIrColor}{HTML}{228B22}
\definecolor{nonLlvmIrColor}{HTML}{B22222}

%Tikz configuration
\usetikzlibrary{arrows.meta, positioning, shapes, snakes, fit, calc}
\tikzset{
    node distance = 8mm and 15mm,
    defaultNode/.style = {rectangle, rounded corners, draw=gray, drop shadow, very thick, bottom color=defaultColor!50!white, text centered, top color=white, minimum height=1.2em, minimum width=3em},
    llvmIrNode/.style = {defaultNode, bottom color=llvmIrColor!50!white},
    nonLlvmIrNode/.style = {defaultNode, bottom color=nonLlvmIrColor!50!white},
    legendNode/.style = {rectangle},
    defaultPath/.style = {-{Stealth[length=4mm]}, color=defaultColor, line width=0.2em, draw, rounded corners},
    llvmIrPath/.style = {defaultPath, color=llvmIrColor},
    nonLlvmIrPath/.style = {defaultPath, color=nonLlvmIrColor}
}

%Custom commands from the according thesis
\newcommand{\opt}{LLVM Optimizer\xspace}
\newcommand{\linker}{LLVM Linker\xspace}
\newcommand{\generator}{LLVM Code Generator\xspace}

\begin{document}

\frame{\titlepage}

\section{Hintergrund}
\subsection{LLVM}
\begin{frame}{History}
\end{frame}
\begin{frame}{Pipeline}
    \begin{figure}[!ht]
%        \caption[A possible pipeline of LLVM]{
%            This visualizes a possible pipeline of \llvm.
%            The reason for a \enquote{possible} pipeline is that the \linker is not necessary when only operating on a single \llvmir file.\tikzlegend
%        }
%        \label{fig:llvmPipeline}
        \centering
        \begin{tikzpicture}
            \node(languages)[nonLlvmIrNode, align=center]{C/C++, Obj-C,\\Python, Ruby,...};
            \node(frontend)[nonLlvmIrNode, right=of languages, align=center]{compiler\\frontend};
            \node(opt1)[llvmIrNode, right=of frontend]{\opt};
            \node(llvmLinker)[llvmIrNode, below=of opt1]{\linker};
            \node(opt2)[llvmIrNode, below=of llvmLinker]{\opt};
            \node(generator)[llvmIrNode, left=of opt2, align=center]{LLVM Code\\Generator};
            \node(linker)[nonLlvmIrNode, left=of generator]{System Linker};
            \path[nonLlvmIrPath] (languages) to (frontend);
            \path[llvmIrPath] (frontend) to (opt1);
            \path[llvmIrPath] (opt1) -| ($(opt1.north east) + (0.5, 0.5)$) -| node[auto, above=0.1cm, left=0.1cm]{Pass} (opt1);
            \path[llvmIrPath] (opt2) |- ($(opt2.south east) + (0.5, -0.5)$) |- node[auto, above=0.1cm]{Pass} (opt2);
            \path[llvmIrPath] (opt1) to node[auto, swap]{Multiple files} (llvmLinker);
            \path[llvmIrPath] (llvmLinker) to node[auto, swap]{Single file} (opt2);
            \path[llvmIrPath] (opt2) to (generator);
            \path[nonLlvmIrPath] (generator) to (linker);
        \end{tikzpicture}
    \end{figure}
\end{frame}
\begin{frame}{Defintion Basic Block}
\end{frame}
\subsection{Polly}
\begin{frame}{History}
\end{frame}
\begin{frame}{Pipeline}
    \begin{figure}[H]
%        \caption[The pipeline of Polly]{
%            The pipeline of Polly \cite{PollyPresentation}.
%            This figure shows the part of the pipeline of \llvm where Polly comes into effect.
%            In this detail the communication between Polly and the \opt is visible.
%            When Polly gets some peace of information it detects the \scops applies its own passes, generates \llvmir and transfers it back to the \opt.
%            A complete pipeline of \llvm is shown in \autoref{fig:llvmPipeline}.\tikzlegend
%        }
        \centering
        \begin{tikzpicture}
            \coordinate(clang);
            \node(opt)[llvmIrNode, right=of clang]{\ac{LLVM} Optimizer};
            \node(polly)[llvmIrNode, below=of opt]{\ac{LLVM} Polly};
            \coordinate[right=of opt](generator);
            \path[llvmIrPath] (clang) to (opt);
            \path[llvmIrPath, bend right] (opt.south west) to node[auto, swap]{SCoP Detection} (polly);
            \path[llvmIrPath, bend right] (polly) to node[auto, swap]{Code Generation} (opt.south east);
            \path[llvmIrPath] (opt) to (generator);
            \path[llvmIrPath] (polly) edge[loop below] ();
            \path[llvmIrPath] (opt) edge[loop above] node[auto]{Pass} ();
        \end{tikzpicture}
        \label{fig:pollyPipeline}
    \end{figure}
\end{frame}
\begin{frame}{Definition Region}
\end{frame}
\begin{frame}{Definition SCoP}
\end{frame}

\section{Empirische Studie}
\subsection{Goals}
\subsection{Measurement Methology}
\begin{frame}{Definition Überdeckung}
\end{frame}
\begin{frame}{Amdahl's Law}
\end{frame}
\begin{frame}{Rejection Reasons}
\end{frame}
\subsection{Tasks}
\subsection{Experiment Setting}

\section{Analyse}
\begin{frame}{\(DynCov_s\) und \(DynCov_p\)}
    \begin{figure}[!h]
        \includesvg[height=\textheight]{compDyncovScopParent}
    \end{figure}
\end{frame}
\begin{frame}{Gründe für Ablehnung}
    \begin{figure}[!h]
        \includesvg[width=1.1\textwidth]{pieInvalidReasons}
     \end{figure}
\end{frame}

\section{Diskussion}

\end{document}
